% imposta header con intestazione "Introduzione"
\markboth{Introduzione}{Introduzione}

% Capitolo non numerato. Non inserito nell'indice
\chapter{Introduzione}

Al giorno d'oggi, anche grazie alla maggiore esposizione mediatica, vi è un termine che sta entrando sempre maggiormente nell'immaginario collettivo: \textit{\textbf{forensics}}

L'etimologia di questo termine inglese, ormai sempre più presente anche nella lingua italiana, deriva dal latino \textbf{\textit{forensis}}. 

Questa espressione stava ad indicare un pubblico dibattito nel quale accusato ed accusatore, coinvolti in un caso giudiziario, proponevano le loro argomentazioni di fronte ad una assemblea (\textit{forum}). 

Con il tempo questo ha portato a riconoscere con il termine \textit{forensics} sia il dibattito pubblico che la presentazione delle prove. Parallelamente all'evoluzione del sistema giudiziario si è potuto assistere anche al progresso delle prove richieste: un processo non si riduceva più ad un semplice dibattito verbale. Anche se permane il concetto del \textit{``di fronte ad una assemblea''}, quello che risulta cambiato è il tipo delle prove presentate e l'importanza che queste hanno nello svolgimento del dibattito \cite{Noctis}. Le evidenze raccolte e la loro analisi sono ormai sempre più vicine ad quella che è la \textbf{\textit{forensic science}}. 

Il termine \textit{forensic science} indica le branche della scienza che hanno il compito di \textbf{riconoscere}, \textit{identificare} e \textit{valutare} le prove fisiche. Divenuta ormai parte integrante del sistema giudiziario, si avvalla di un'ampia gamma di scienze per ottenere informazioni rilevanti e utili. Lo scopo della \textit{forensic science} può essere riassunto come \textit{``applicare metodologie scientifiche e processi per la risoluzione di un caso''} \cite{Forensics}.

Fra le branche della \textit{forensic science} si trova la \textbf{\textit{computer forensics}}. Questa si prefigge l'andare oltre la semplice ricerca ed estrazione dei dati. Cerca di analizzare in maniera oggettiva ciò che è stato estratto, in modo da non privilegiare una parte o l'altra, provando ad essere \textit{super partes} e andando più in profondità alla ricerca di quello che potrebbe non risultare visibile ad una prima analisi. Il tutto seguendo principi di ``pertinenza'', legata alla ricerca delle informazioni che realmente dimostrino valore probatorio.

Molti sono oggi gli atti criminali che vengono preparati e realizzati mediante informazioni reperibili sulla rete Internet o con l'ausilio di sistemi informatici. Riuscire a ricostruire queste informazioni o perlomeno fornire alle autorità inquirenti degli indizi da seguire, potrebbe rivelarsi molto utile e cambiare le sorti di un dibattito giudiziario.

L'intento di questo lavoro è quello di studiare ed analizzare il contenuto della \textbf{\textit{memoria cache}} dei vari Web Browser, in modo da comprendere come gli stessi memorizzano e gestiscono le informazioni presenti al suo interno. 

Successiva a questa fase di studio vi è lo sviluppo di un'applicazione che individuati i Web Browser installati, permette l'estrazione dei dati presenti nella memoria cache con il fine di ottenere risultati interpretabili in maniera corretta.

Nel capitolo 1 vengono affrontate le fasi basilari che portano all'ottenimento di una prova, passando per quelle che sono le \textit{best practices} da seguire durante le fasi di acquisizione in modo da non deteriorarne la loro valenza probatoria. 

Il capitolo 2 fornisce una panoramica su quella che è la memoria cache di un Web Browser, sul suo funzionamento ed i vantaggi apportati durante la navigazione Web ed infine sulla sua utilità all'interno di un'indagine investigativa. 
\clearpage

Il capitolo 3 passa in rassegna la memoria cache di ogni Web Browser preso in esame, illustrandone il sistema di memorizzazione dei dati e come questi vengono interpretati per il reperimento delle informazioni contenute.