\chapter{Concetti preliminari: Computer Forensics}

Nel quotidiano, dalla sfera lavorativa a quella sociale, l'informatica assume un ruolo sempre più preponderante. Di pari passo, anche per quanto riguarda i reati e la giustizia, sorgono nuove problematiche. Come valutare i vari dispositivi informatici, ormai sempre più presenti ed importanti durante un caso giudiziario? Come procedere alla raccolta e conservazione delle prove informatiche individuate? Quale impatto hanno durante lo svolgimento di un caso giudiziario? \nocite{Florindi}

\section{La prova digitale}

Un dispositivo elettronico o informatico (smartphone, tablet, personal computer) può essere visto come \textbf{prova} se considerato nella sua interezza, o come \textbf{mezzo per la ricerca della prova}. Quello che realmente interessa non è il dispositivo in senso fisico ma in senso ``logico''. \`E il suo \textbf{contenuto} che potrebbe risultare rilevante dal punto di vista probatorio ed assumere significati diversi sulla base delle evidenze che si stanno cercando, come quelle \textbf{fisiche} (hardware) oppure \textbf{logiche} (software e dati). L'organizzazione SWGDE \footnote{Scientific Working Group on Digital Evidence \url{https://www.swgde.org/}} definisce una prova digitale come una \textit{``\textbf{informazione dal valore probatorio, memorizzata o trasmessa in formato digitale}''} \cite{Swdge}.

Diverso è anche l'approccio utilizzabile in sede di analisi. Si può passare da un profilo di tipo ``tecnico'' in cui si procede alla mera estrazione dei dati senza alcuna interpretazione degli stessi, ad un altro di tipo ``investigativo'' in cui si procede anche ad un'analisi degli stessi per cercare di ricostruire il \textit{modus operandi} dell'utente. Il profilo da tenere viene indicato dal soggetto conferente l'incarico, avendo però cura di tenere separate le risultanze oggettive dalle eventuali deduzioni.

Un tema delicato che molte volte vede contrapposte accusa e difesa riguarda il \textbf{rilevamento}, la \textbf{conservazione} ed il \textbf{trattamento} del materiale costituente la prova da esaminare. Per questo è auspicabile l'utilizzo di un protocollo che ne garantisca la corretta integrità ed il non ripudio in sede processuale. \`E su questo materiale raccolto ed analizzato che si baserà la valutazione dell'organo giudicante ricordando che la loro alterazione è molto spesso un processo irreversibile.

\section{Acquisizione della prova digitale}

Diverse sono le tipologie di crimine e diverse sono le tracce lasciate dagli autori dell'illecito. Ad esempio, sulla base di queste, possiamo distinguere fra \textbf{reati permanenti} (ad esempio la realizzazione di un sito web) e \textbf{reati istantanei} (vedasi ingiurie in chat o condivisione di materiale illecito).

Le modalità da seguire nelle indagini devono risultare versatili in quanto diversi sono i tipi di reati e gli strumenti forniti dal legislatore.

Alcune caratteristiche però risultano comuni alle varie tipologie di indagini. Fasi come quella dell'\textbf{acquisizione}, \textbf{conservazione}, \textbf{trattamento}, \textbf{analisi}, \textbf{esposizione} e \textbf{relazione} delle stesse, risultano facenti parte di una catena di eventi che non possono essere slegati l'uno dall'altro. Soprattutto nelle evidenze informatiche, causa la loro volatilità, la fase di acquisizione deve essere il più celere possibile, avendo particolare premura, in questa fase, di evitare alterazioni e danneggiamenti sia volontarie che involontarie. Intercettazioni, perquisizioni ed ispezioni devono rispettare protocolli procedurali al fine di evitare vizi che andrebbero ad invalidarne la validità processuale. Va ricordato anche che in questa fase l'indagato può risultare all'oscuro di indagini a suo carico, non sussistendo qui l'obbligo di avvisarlo della sua posizione.

\section{Indagini difensive}

Lo status di imputato o di indagato non è necessario per permettere lo svolgimento di indagini difensive svolte in maniera preventiva. Se si sospetta di essere oggetto di un procedimento nei propri confronti, si può incaricare un difensore di fiducia per acquisire elementi utili ad un'eventuale futura difesa.

Importante è, soprattutto in presenza di reati tecnologici, che l'avvocato si avvalga di consulenti di fiducia in grado di indirizzarlo verso una corretta indagine difensiva. Un consulente, come un investigatore, può essere nominato in qualsiasi momento, preferibilmente formalizzando la nomina. In questo modo si permette allo stesso di apporre il segreto professionale alle domande degli inquirenti.

L'investigatore, dal canto suo, necessita di competenze idonee al lavoro con le delicate evidenze informatiche per non danneggiarle o inquinarle durante le attività di raccolta e analisi. 

\section{Perquisizione}

Il soggetto scopre di essere indagato alla ricezione di una ``informazione di garanzia'' che molto spesso, per quello che riguarda i reati informatici, arriva insieme ad un decreto di perquisizione e sequestro.

Per un soggetto è possibile subire una perquisizione anche senza avere lo status di indagato: esempio tipico è la perquisizione presso soggetti terzi nel caso di indagini a carico di ignoti, oppure nel caso in cui l'autore del reato non sia il proprietario del bene sequestrato (esempio del dipendente che commette reato da dispositivi informatici aziendali). Si tratta di scenari sempre più frequenti e si apre, per maggiore correttezza giuridica, un procedimento contro ignoti. L'alternativa sarebbe iscrivere nel registro degli indagati il titolare dell'utenza telefonica: individuato a partire dall'indirizzo IP utilizzato per commettere reato, potrebbe ritrovarsi ad essere indagato semplicemente per il fatto di essere il titolare dell'utenza. Questo però risulterebbe scorretto in quanto si avrebbe una responsabilità penale che a differenza di quella civile, che indica varie forme di responsabilità per fatto altrui, è totalmente soggettiva. Nessuno quindi dovrebbe essere indagato, almeno fino alla perquisizione. Il procedimento dovrebbe avere inizio contro ignoti e poi, con il prosieguo delle indagini, capire chi ha commesso il reato ed iscriverlo nel registro degli indagati.  

La perquisizione rappresenta un momento molto delicato nella fase delle indagini. Si consideri il caso di reperti non individuati in prima istanza, probabilmente destinati a sparire successivamente e non essere più utilizzabili. Per questo motivo sarebbe buona norma non fermarsi solo agli elementi facilmente visibili, cercando con particolare cura anche quei dettagli che potrebbero poi rivelarsi molto importanti.

Nelle perquisizioni che, salvo casi di urgenza disposti dall'autorità giudiziaria, devono essere svolte fra le ore 7:00 e le ore 20:00 \footnote{Art. 251 c.p.p.}, è anche possibile farsi assistere da una persona di fiducia (purchè idonea e prontamente reperibile).

\subsection{Best practices}

In maniera ideale, si dovrebbe provvedere a ``cristallizzare'' la scena contenente le prove, evitandone ogni contaminazione.
I primi accorgimenti, da tenere non appena giunti su una scena da perquisire, sono quelli che avranno poi un maggiore riflesso sul corretto svolgimento delle successive fasi. Errori commessi in questa fase possono ripercuotersi successivamente. Le regole basilari indicano di: 

\begin{itemize}
	
	\item{evitare di accendere i computer trovati spenti} 
	
	\item{decidere se e come spegnere quelli trovati accesi}
	
	\item{annotare con cura, anche documentando con fotografie, la disposizione degli apparati all'interno della scena} 

\end{itemize}

Come comportarsi quindi? In caso di computer spento, si esegue un'immagine dell'hard disk acquisendone anche la relativa firma \textit{hash} annotandola sul verbale. Sarebbe buona norma controllare che la copia effettuata sia leggibile e che non vi siano dispositivi che inibiscano la lettura sull'hard disk del computer in esame. In caso di dubbi risulta conveniente non limitarsi alla copia ma prelevare tutto il materiale, illustrandone le ragioni nel verbale di sequestro.

In caso di computer acceso lo scenario diventa più complesso. La procedura corretta sarebbe poter disporre di un soggetto competente che, alla presenza delle parti, esegua l'analisi della macchina accesa annotando le operazioni svolte e documentando i processi e programmi in esecuzione sulla stessa, per poi procedere ad uno spegnimento sicuro.

Se tutto questo non fosse possibile, la soluzione meno invasiva prevede il distacco diretto della spina della corrente (e la rimozione della batteria nel caso di un portatile). In questo modo si perde il contenuto della memoria RAM, ma risulta la procedura più sicura per personale poco esperto.

In caso invece di macchine virtuali in esecuzione, la procedura richiede una immediata ispezione in loco e la conseguente acquisizione del contenuto del disco e della macchina virtuale, in quanto molto probabile che queste contengano password richieste all'avvio e che l'imputato può legittimamente rifiutare di concedere.

\subsection{Sequestro o copia}

Un tipico scontro fra accusa e difesa durante le fasi di un processo avviene riguardo le modalità con le quali si procederà al sequestro delle evidenze. \cite{Vierika} In caso di reato informatico ci si chiede se sia necessario, ad esempio, sequestrare l'intero computer oppure se basti acquisire una copia dei dati. Da non trascurare è anche il materiale ritenuto non di primaria importanza, come le varie periferiche di un personal computer, le quali possono contenere elementi ed informazioni utili (impronte digitali, password o altre annotazioni). Va comunque ricordato che a diversi reati corrispondono differenti tipologie di indagine e differenti metodi di acquisizione della prova. Inoltre, a parità di risultato probatorio, dovrà essere sempre privilegiata la modalità meno invasiva per chi subisce la perquisizione, soggetto che a volte può anche non essere un indagato ma solo una ``persona informata sui fatti''. Importante rimane anche l'ipotesi che un mancato sequestro di alcune apparecchiature, anche quelle ritenute secondarie,  potrebbe pregiudicare la possibilità di svolgere ulteriori accertamenti. Collocazione, stato del computer e accessi presenti possono fornire informazioni utili sulla personalità del presunto reo.

La modalità meno invasiva prevederebbe di effettuare una copia dei dati direttamente in sede di perquisizione. Questo scenario ha in sè due problematiche:

\begin{itemize} 

	\item{richiede la presenza di personale esperto in grado di operare in maniera sicura su supporti hardware e software sconosciuti e di individuare, in tempi brevissimi, tutti i file di interesse probatorio (compresi quelli nascosti, cancellati, crittografati o steganografati).}

	\item{l'operazione, anche se eseguita secondo le \textit{best practices} della \textit{computer forensics}, è di fatto irripetibile, in quanto il materiale rimasto a disposizione dell'imputato deve considerarsi non più utile per finalità investigative (contaminato).}

\end{itemize}

Un buon compromesso è rappresentato dal sequestro del solo hard disk (oppure di un'acquisizione, con strumenti idonei, di un'immagine dello stesso se si ritiene opportuno lasciare l'hard disk nella disponibilità dell'imputato).
Tale soluzione, applicabile alla maggior parte dei reati informatici, consente un pieno controllo del contenuto del supporto e la ripetibilità, in qualsiasi momento, dell'analisi eseguita. Richiede però la presenza, in sede di perquisizione e sequestro, di personale esperto in grado di rimuovere l'hard disk.

La perquisizione con la conseguente acquisizione dei soli file pertinenti al reato contestato andrebbe preferita nei casi in cui il computer può essere considerato come un ``contenitore'' della prova. Vi sono situazioni in cui però è necessario procedere al sequestro dell'intero computer e delle relative periferiche, come nel caso di indagini pedopornografiche. In questi casi non si parlerà di sequestro probatorio \footnote{Art. 253 c.p.p.} ma di sequestro preventivo \footnote{Art. 321 c.p.p.}. \`E anche possibile che una copia dell'hard disk venga effettuata direttamente in sede di sequestro, soprattutto in caso di sequestro probatorio. 
 
\subsection{Cloni}

Il principio alla base della \textit{computer forensics} è quello della \textit{ripetibilità dell'analisi}. Non sempre in fase di acquisizione è possibile rispettarlo. Sulla base della natura dell'accertamento eseguito questo può essere classificato come \textbf{ripetibile} o \textbf{irripetibile} 

Un accertamento ripetibile può essere eseguito senza alcuna comunicazione. La difesa poi, in sede processuale, può richiederne la ripetizione alla presenza delle parti e di un perito. 

L'accertamento non ripetibile suppone che possano esserci modifiche ed alterazioni. Per questo motivo il Pubblico Ministero avvisa la parte offesa e la parte indagata sul giorno, ora e luogo del conferimento dell'incarico.

Senza dubbio sarebbe da preferirsi un accertamento ripetibile, così come è sconsigliato l'utilizzo del materiale originale, onde evitarne l'alterazione o il danneggiamento del supporto contenente. 
Implicito è che errori o altro che apporti modifiche al reperto ne facciano decadere la validità in ambito dibattimentale, facendolo risultare ``inquinato''. La questione riguarda il giusto processo in quanto la qualificazione di un atto come ``irripetibile'' comporta la deroga del \textbf{principio del contraddittorio} nella formazione della prova. Rimane comunque possibile per le parti far valere eventuali nullità relative all'osservanza delle forme previste, nel caso in cui i documenti acquisiti vengano utilizzati come prova.

Ecco perchè, in quanto alle evidenze informatiche, si utilizzano copie esattamente conformi del supporto fisico sul quale risiedono. In gergo si parla di ``clonazione'', la quale permette di ottenere un disco perfettamente identico all'originale e sul quale effettuare le analisi del caso, motivo per il quale è importante assumere tutte le protezioni necessarie a ridurre il rischio di danneggiamenti del supporto o del suo contenuto. Ad esempio, nell'occorrenza di dover avviare il dispositivo informatico in sede di perquisizione, si provvede ad utilizzare un dispositivo \textbf{write blocker}, il quale inibisce la scrittura sul disco non consentendo modifiche indesiderate. La metodologia più sicura rimane comunque l'avvio del sistema in modalità \textbf{virtuale} dove non si corre rischio di danneggiare dati e programmi presenti.

Buona norma è l'acquisizione, precedente e successiva alla clonazione, della firma \textit{hash} del supporto. Stampando la firma elettronica, generata con algoritmi MD5 o SHA e facendola sottoscrivere alla controparte, è possibile dimostrare la non avvenuta compromissione durante la creazione della copia e che sorgente e  risultino perfettamente identiche e conformi.

\section{Analisi}

Ultima fase di questo lungo processo partito dalla ricerca delle evidenze probatorie è l'analisi delle stesse.
Il personale addetto dovrebbe possedere una elevata competenza e prestare particolare cura ai reperti in fase di analisi, in quanto errori in questo momento ne pregiudicherebbero la valenza probatoria.

In questa fase non è importante verificare la sola presenza di materiale illecito quanto invece se quel materiale sia stato \textbf{consapevolmente} acquisito, ceduto a terzi o trattato in altri modi. Ad esempio è possibile procedere a questo accertamento confrontando la presenza di determinati files con le tracce lasciate da una loro ricerca mediante un motore di ricerca.

A rendere più difficoltose le analisi possono essere protezioni, steganografia, crittografia o la non presenza fisica sulla macchina ma su \textit{client} di posta elettronica o \textit{cloud}.

Crittografia e steganografia sono perfettamente legali, così come legale è il possibile rifiuto da parte dell'imputato di fornire le credenziali per l'accesso a tali risorse, valutabile però negativamente dal giudice in caso di condanna.

Contestazioni possono verificarsi sull'utilizzo dei software per le analisi o sugli analisti che le effettuano. Nel primo caso possono venir contestati \textit{software proprietari} in quanto non è possibile sapere come si è arrivati all'ottenimento di un determinato risultato data la natura \textit{closed source} del programma utilizzato. 

Nel secondo caso invece ad essere sotto la lente non sono gli strumenti utilizzati ma il tecnico e, principalmente, il suo bagaglio legale e professionale. Fondamentali infatti in questo tipo di analisi risultano preparazione e scrupolosità del tecnico incaricato. 
