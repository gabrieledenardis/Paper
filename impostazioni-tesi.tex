% ********************************************************************
% Packages da utilizzare
%******************************************************* 

	% encoding ad 8 bit per fonts
	\usepackage[T1]{fontenc}

	% inserimento diretto lettere accentate
	\usepackage[utf8]{inputenc}

	% cambia il reset del contatore delle note
	\usepackage{chngcntr}

	% no reset numero del footnote ad ogni capitolo
	\counterwithout{footnote}{chapter}

	% miglioramenti per inserimento figure e tabelle
	\usepackage{float}
	
	% disegno directory tree
	\usepackage{dirtree}

	% lingua principale del documento e sillabazione
	\usepackage[italian]{babel}

	% lingua bibliografia corrispondente a quella del documento
	\usepackage{babelbib}

	% gestione colori (dvipsnames evita problemi se usato con pdfpages, table per righe colori alternati tabelle)
	\usepackage[dvipsnames,table]{xcolor}

	% inserimento documenti pdf
	\usepackage{pdfpages}

	% layout itemize, enumerate e descrizioni
	\usepackage{enumitem}

	% gestione footnote
	\usepackage{footnote}

	% supporto esteso per grafica
	\usepackage{graphicx}

	% inserimento url (hyphens per spezzare url in corrispondenza del trattino)
	\usepackage[hyphens]{url}

	% abilitazione riferimenti ipertestuali (no bordi intorno link, no voci indice link, numeri pagina indice link)
	\usepackage[linktocpage]{hyperref}

	% stile classicthesis
	\usepackage[eulerchapternumbers,beramono,eulermath,pdfspacing]{classicthesis}     
	          
	% uso del package arsclassica. Contiene modifiche al pacchetto classicthesis
	\usepackage{arsclassica}    
	
	% gestione dimensione pagina (a4paper = gestione per formato a4)
	\usepackage[a4paper]{geometry}
	
	% altezza del campo del testo
	%\textheight = 625pt
	
	% definizione layout introduzione
	\newcommand\layoutintroduzione{
		
		% cancellazione header e footer predefiniti classicthesis
		\clearscrheadfoot
		
		% header personalizzato introduzione. ohead = lehead,rohead
		\ohead{\textit{Introduzione}}
		
		% numero pagina al centro per tutte le pagine (pari-dispari)
		\cfoot[\pagemark]{\pagemark}
	}

	% definizione altri capitoli
	\newcommand\layoutaltricapitoli{

		% cancellazione header e footer predefiniti classicthesis
		\clearscrheadfoot
		
		% cancellazione header e footer predefiniti classicthesis
		\clearscrheadfoot
		
		% titolo del capitolo (destra-pagine dispari)
		\rohead{\textsc{\thechapter - \leftmark}}
		
		% titolo personalizzato (sinistra-pagine pari)
		\lehead{\textsc{Browser Forensic: analisi della cache dei principali browser}}
		
		% numero pagina al centro per tutte le pagine (pari-dispari)
		\cfoot[\pagemark]{\pagemark}
	}               

% ********************************************************************
% Impostazioni
%*******************************************************

	% percorso (relativo) cartella immagini
	\graphicspath {{./Immagini/}}

	% interlinea caratteri
	\linespread{1.5}
	
	% numerazione sezioni fino al livello 3
	\setcounter{secnumdepth}{3}
	
	% numero di sottosezioni nell'indice
	\setcounter{tocdepth}{3}

